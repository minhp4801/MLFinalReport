
%% The first command in your LaTeX source must be the \documentclass command.
\documentclass[sigconf]{acmart}

%% \BibTeX command to typeset BibTeX logo in the docs
\AtBeginDocument{%
  \providecommand\BibTeX{{%
    \normalfont B\kern-0.5em{\scshape i\kern-0.25em b}\kern-0.8em\TeX}}}

\begin{document}

\title{Recurrent Adversarial Audio Synthesis}

\author{Bryan Medina}
\email{bryanjosemedina@knights.ucf.edu}
\affiliation{%
    \institution{}
}
\author{Minh Pham}
\email{minh.pham@knights.ucf.edu}
\affiliation{%
    \institution{University of Central Florida}
    \city{Orlando}
    \state{Florida}
}
\author{Kobee Raveendran}
\email{kobee.raveendran@knights.ucf.edu}
\affiliation{%
    \institution{}
}
%%
%% By default, the full list of authors will be used in the page
%% headers. Often, this list is too long, and will overlap
%% other information printed in the page headers. This command allows
%% the author to define a more concise list
%% of authors' names for this purpose.
\renewcommand{\shortauthors}{Trovato and Tobin, et al.}

%%
%% The abstract is a short summary of the work to be presented in the
%% article.
\begin{abstract}
  
\end{abstract}

\keywords{datasets, neural networks, gaze detection, text tagging}

\maketitle

\section{Introduction}
\section{An Argument for GRUs}

\begin{acks}

\end{acks}


\bibliographystyle{ACM-Reference-Format}
\bibliography{sample-base}

%%
%% If your work has an appendix, this is the place to put it.
\appendix
\end{document}
\endinput
%%
%% End of file `sample-sigconf.tex'.
